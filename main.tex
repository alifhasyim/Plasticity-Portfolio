\documentclass[12pt]{article}
\usepackage{graphicx}
\usepackage{caption}
\usepackage{amsmath}
\usepackage{geometry}
\usepackage{float}
\usepackage{booktabs}
\usepackage{longtable}
\usepackage{setspace} % for spacing control
\usepackage{indentfirst}
\usepackage{titlesec}
\setstretch{1.0}       % single line spacing (default)
\setlength{\parskip}{0pt}   % no space between paragraphs
\setlength{\parindent}{0em} % set indentation amount

\geometry{margin=1in}
\setlength{\parskip}{0em}
\setlength{\parindent}{0pt}

\title{Self-Evaluation Report \\ \large Computational Plasticity (SoSe25)}
\author{Bagus Alifah Hasyim \\ 108023246468}
\date{}

\begin{document}
\maketitle

\section*{Phenomenology of Plasticity}

\section{Difference between elastic and plastic strain}
\hspace{2em}To understand the difference between elastic and plastic strain, we can start by visualizing a simple stress-strain
curve. 
\section{Microscopic origin of plastic deformation}

\section*{Continuum Plasticity}
\section{Equivalent stress and yield function}
\section{Different yield criteria}
\section{Evolution of yield locus with plastic strain}
\section{Associated flow rule}

\section*{Crystal Plasticity}
\section{Orowan law}
\section{Resolved shear stress}
\section{Finite strain formulation}
\section{Crystal plasticity flow rule}

% Example citation
According to \cite{Hill1950}, the theory of plasticity provides a framework for understanding material behavior beyond elastic limits.

\bibliographystyle{plain}
\bibliography{references}

\end{document}

