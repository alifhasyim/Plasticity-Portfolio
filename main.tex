\documentclass[12pt]{article}
\usepackage{graphicx}
\usepackage{caption}
\usepackage{amsmath}
\usepackage{geometry}
\usepackage{float}
\usepackage{booktabs}
\usepackage{longtable}
\usepackage{setspace} % for spacing control
\usepackage{indentfirst}
\usepackage{titlesec}

\setstretch{1.0}       % single line spacing (default)
\setlength{\parskip}{0pt}   % no space between paragraphs
\setlength{\parindent}{0em} % set indentation amount

\geometry{margin=1in}
\setlength{\parskip}{0em}
\setlength{\parindent}{0pt}

\title{Self-Evaluation Report \\ \large Computational Plasticity (SoSe25)}
\author{Bagus Alifah Hasyim \\ 108023246468}
\date{}

\begin{document}
\maketitle

\section*{Phenomenology of Plasticity}

\section{Difference between elastic and plastic strain}
\hspace{2em}To understand the difference between elastic and plastic strain, we can start by visualizing a simple stress-strain
curve. 

\hspace{2em}
As the material loaded with some external force, it will start to deform elastically. Here, the relation
between the stress (Y-Axis) and strain (X-Axis) shown a linear behavior until it reach a point where the materials start to yield.
The distance between the 0 point and yield point in X direction is called elastic strain. In a microscopic sense, the atoms are
displaced slightly in the direction of loading, which here there will be no displacement in between the atom. The bond, such 
as grain boundaries and atomic orientation will remain unchanged in this phase.

\hspace{2em}
When the material exceeds the yield point, the deformation behavior will change. The material will not deformed elastically anymore, 
but rather plastically, in a microscopic sense that there will be some dislocation or a significant move in between the atomic bond
inside the material. Hence, this stage generates a non-linear stress-strain relationship, by means that the material will experience
a permanent deformation, in other words a non-reversible process. As the material unloaded, the loading curve will be reversed, which
the slope of the line is equal to the slope at the elastic region. When it reaches zero, this is the plastic strain point. Where, the 
plastic strain is measure from the yield point and the plastic strain point.
\section{Microscopic origin of plastic deformation}
\hspace{2em}
To elaborate more detail about the microscopic phenomenology on plastic deformation, explaining the key idea on microscopic 
scale of material should be done beside understanding it on a stress-strain curve. In general for a single crystal deformation, 
the loading will create a shear strain with respect to the crystal slip plane over another. This induced shear stress will cause
then will create a relative lateral movement from both parallel planes~\cite{taylor1934plastic}. 

\hspace{2em}
Based on theories of the equilibrium of crystal lattices, an atom should place itself in a minimum potential energy position
with respect to its neighbors. Whenever the atom moves, it will create heat internally. These minimum potential energy
will create "potential barriers", that restrict the atom to dislocate in between planes. As the dislocation exceed the
potential barrier, the atoms neighborhoods will rearrange themselves by a continuous process so that it will create a 
new equilibrium energy~\cite{taylor1934plastic}. This will create a new energy barriers, and the atoms position will not go back to
its origin since it has a new potential barrier. To introduce the same behavior again regarding the atom dislocation, 
one should apply condition where it will dislocate the atoms until it reach the updated minimum potential barriers.

\section*{Continuum Plasticity}
\section{Equivalent stress and yield function}
\hspace{2em}
The idea of equivalent stress is that there are needs to describe stress state in a unified or a scalar quantity
that can be compared to other quantities such as yield strength. Because as the material loaded by some external forces or traction,
it can introduce stress generation that has variation in terms of direction. To represent the stress component in each
direction, equivalent stress is used to model the stress state in the material. Because from the analysis, the stress result
can be only compared with material data that is obtained through an uniaxial tensile test. Here, the data only describe
the material flow in one direction of loading. To describe the stress in more detail, one can write the stress component as follows:
\begin{equation}
\boldsymbol{\sigma} = 
\begin{bmatrix}
\sigma_{xx} & \sigma_{xy} & \sigma_{xz} \\
\sigma_{yx} & \sigma_{yy} & \sigma_{yz} \\
\sigma_{zx} & \sigma_{zy} & \sigma_{zz}
\end{bmatrix}
\end{equation}

We introduce the decomposition of the stress, hence consist of hydrostatic and deviatoric component:
\begin{equation}
\boldsymbol{\sigma} = \boldsymbol{\sigma}_{hydro} + \boldsymbol{\sigma}_{dev}
\end{equation}

Where the hydrostatic stress is the stress component that cause a volumetric change of the material and 
the deviatoric stress component will cause shape change, which here for instance a distortion in the material. The
hydrostatic stress can be defined as follow:
\begin{equation}
\boldsymbol{\sigma}_{hydro} = \frac{1}{3} \text{tr}(\boldsymbol{\sigma}) \boldsymbol{I} = \frac{1}{3} \left( \sigma_{xx} + \sigma_{yy} + \sigma_{zz} \right) \boldsymbol{I}
\end{equation}

Then we are interested in the deviatoric stress component, which can be defined as:
\begin{equation}
\boldsymbol{\sigma}_{dev} = \boldsymbol{\sigma} - \boldsymbol{\sigma}_{hydro}
\end{equation}

By inserting the full stress component, we can write the deviatoric stress component as a full stress component in a matrix component
In this sense:
\begin{equation}
\boldsymbol{\sigma}_{dev} =
    \begin{bmatrix}
    \sigma_{xx} - \frac{1}{3}\text{tr}(\boldsymbol{\sigma}) & \sigma_{xy} & \sigma_{xz} \\
    \sigma_{yx} & \sigma_{yy} - \frac{1}{3}\text{tr}(\boldsymbol{\sigma}) & \sigma_{yz} \\
    \sigma_{zx} & \sigma_{zy} & \sigma_{zz} - \frac{1}{3}\text{tr}(\boldsymbol{\sigma}) \\
    \end{bmatrix}
    = s_{ij}
\end{equation}

From this we introduce the second invariant of the deviatoric stress, which is defined as:
\begin{equation}
J_2 = \frac{1}{2}s_{ij}s_{ij} = \frac{1}{2}[\text{tr}(s)^2-\text{tr}(s^2)]
\end{equation}

We calculate each of the trace component in the J2 invariant:
\begin{equation}
\text{tr}(s) = s_{11} + s_{22} + s_{33} = 0
\end{equation}
\begin{equation}
\text{tr}(s^2) = s_{11}^2 + s_{22}^2 + s_{33}^2 + 2(s_{12}^2 + s_{13}^2 + s_{23}^2)
\end{equation}

\section{Different yield criteria}
\section{Evolution of yield locus with plastic strain}
\section{Associated flow rule}

\section*{Crystal Plasticity}
\section{Orowan law}
\section{Resolved shear stress}
\section{Finite strain formulation}
\section{Crystal plasticity flow rule}

% Example citation
According to \cite{hill1950}, the theory of plasticity provides a framework for understanding material behavior beyond elastic limits.

\bibliographystyle{plain}
\bibliography{bibliography_CIBB_file.bib}

\end{document} 

